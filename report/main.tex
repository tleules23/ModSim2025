\documentclass[a4paper,11pt]{scrartcl}

\usepackage[english]{babel}
% uncomment or modify according to your operating system
% \usepackage[applemac]{inputenc} % european characters can be used (Mac OS)
%\usepackage[latin1]{inputenc} 
%\usepackage{ucs} % fuer deutsche Sonderzeichen unter Linux
%\usepackage[utf8x]{inputenc} %fuer deutsche Sonderzeichen unter Linux
\usepackage[utf8]{inputenc} %fuer deutsche Sonderzeichen unter Windows
\usepackage[T1]{fontenc}
\usepackage{graphicx} 
\usepackage{fullpage}
\usepackage{latexsym}
\usepackage{amssymb}
\usepackage{amsmath}
\usepackage{ifthen}
\usepackage{listings}
\usepackage{color}
\usepackage{hyperref}
\usepackage{cite}

\definecolor{dkgreen}{rgb}{0,0.6,0}
\definecolor{gray}{rgb}{0.5,0.5,0.5}
\definecolor{mauve}{rgb}{0.58,0,0.82}
\lstset{ %
  language=Matlab,                % the language of the code
  basicstyle=\small,              % the size of the fonts that are used for the code
  numbers=left,                   % where to put the line-numbers
  numberstyle=\tiny\color{gray},  % the style that is used for the line-numbers
  stepnumber=1,                   % the step between two line-numbers. If it's 1, each line 
                                  % will be numbered
  numbersep=5pt,                  % how far the line-numbers are from the code
  backgroundcolor=\color{white},      % choose the background color. You must add \usepackage{color}
  showspaces=false,               % show spaces adding particular underscores
  showstringspaces=false,         % underline spaces within strings
  showtabs=false,                 % show tabs within strings adding particular underscores
  frame=single,                   % adds a frame around the code
  rulecolor=\color{black},        % if not set, the frame-color may be changed on line-breaks within not-black text (e.g. commens (green here))
  tabsize=2,                      % sets default tabsize to 2 spaces
  captionpos=b,                   % sets the caption-position to bottom
  breaklines=true,                % sets automatic line breaking
  breakatwhitespace=false,        % sets if automatic breaks should only happen at whitespace
  title=\lstname,                   % show the filename of files included with \lstinputlisting;
                                   % also try caption instead of title
  keywordstyle=\color{blue},          % keyword style
  commentstyle=\color{dkgreen},       % comment style
  stringstyle=\color{mauve},         % string literal style
  escapeinside={\%*}{*)},            % if you want to add LaTeX within your code
  morekeywords={end,sortrows}               % if you want to add more keywords to the set
}


% Abkuerzungen fuer haeufige Befehle wie z.B. griechische Buchstaben
\newcommand{\N}{\mathbb{N}}
\newcommand{\Z}{\mathbb{Z}}
\newcommand{\R}{\mathbb{R}}
\newcommand{\C}{\mathbb{C}}
\newcommand{\F}{\mathcal{F}}
\newcommand{\La}{\mathcal{L}}
\newcommand{\de}{\delta}
\newcommand{\e}{\varepsilon}
\newcommand{\la}{\lambda} 
\newcommand{\p}{\varphi} 
\newcommand{\al}{\alpha} 
\newcommand{\be}{\beta} 
\newcommand{\om}{\omega}
\newcommand{\Om}{\Omega}
\newcommand{\ta}{\tau}
\newcommand{\g}{\gamma}
\newcommand{\HE}{\mathbb{H}}
\newcommand{\E}{\mathbb{E}}
\newcommand{\vG}{\varGamma}
\newcommand{\s}{\sigma}

\begin{document}

\subject{194.076 Modeling and Simulation}
\title{Discrete Event Simulation Case Study: Dining Philosophers
Problem}

%  TODO: add names
%  TODO: add description of which parts did woh in the footnote as in the template!

\publishers{Supervisor: Madlen Martinek}
\author{Assylbek Tleules, 12432843 - Curriculum Number: UE 066 645 \footnote{TODO: ADD DESCRIPTION}\\
Arian Ahmad, 01529233 - Curriculum Number: UE 066 926 \footnote{TODO: ADD DESCRIPTION}\\
Hasan Hüseyin Günes, 12229237 - Curriculum Number: UE 066 645\footnote{TODO: ADD DESCRIPTION}\\
}

\maketitle
% TODO
\begin{abstract}
TODO: Brief Abstract
\end{abstract}

\newpage

\tableofcontents

\newpage

\section{Introduction}

When multiple processes or systems share a limited set of resources, several problems can occur. If the coordination is done poorly, the whole system can get stuck in a deadlock, where no further progress is possible. In addition, if the system cannot ensure a certain fairness level, individual components may experience very long waiting times compared to others, also called starvation. The Dining Philosophers problem\footnote{\url{https://en.wikipedia.org/wiki/Dining_philosophers_problem}} is a simple but realistic example of such situations. It helps to understand why these issues can occur and provides a useful basis for copmaring different methods that aim to prevent them. 

As shown in Figure~\ref{fig:diningphilosophers}, in the Dining Philosophers problem, philosophers sit around a table with one chopstick placed between each pair of neighboring philosophers. Over time, each philosopher repeats the cycle of thinking, being hungry and eating. To eat, a philosopher needs two chopsticks (the left and right one), while one chopstick can only be held by one philosopher at a time. Since neighboring philosophers share one chopstick, such conflicts can occur. Depending on the rule for picking up chopsticks, a situation could arise where every philosophers holds one chopstick and waits for the second one to become available. Another issue regarding the fairness could be that some philosophers may wait for much longer then others, when they want to eat.

\begin{figure}[!ht]
 \centering
 \includegraphics[width=0.5\textwidth]{report/images/dining_philosophers_2.png}
 \caption{Dining philosophers sitting at a round table, each with a bowl of food and chopsticks.\protect\footnotemark}
 \label{fig:diningphilosophers}
\end{figure}
\footnotetext{\url{https://diningphilosophers.eu/}}

Based on a discrete event simulation model, this work aims to implement the Dining Philosophers system and analyze its behavior under different configurations and rules. First, the base concept without any additional methods is inspected. Then, methods are introduced to the concept, which aim to detect and prevent deadlocks and to reduce starvation cases. Finally, we consider further scenarios such as varying the number of philosophers and allowing rule violations by individual philosophers. The different settings are compared using the simulation outcomes.  

\section{Model}

TODO: do not forget to update requirements.txt after finishing implementation!

\subsection{Modeling Approach}

describe in general how we modelled (which classes / entitites etc.)

+ f.e. explain 1 time unit = 1 min etc.

\subsection{Base Model}

1. Describe basic model (f.e. includig initial setup of model) 

\subsection{Deadlock Detection / Prevention}

2. explain Deadlock -> describe our Deadlock Detection / Prevention Concepts

\subsection{Starvation Avoidance}

3. explain Starvation -> describe our Starvation Avoidance Concept

Deadlocks are not the only issues that can occur during the simulation runs. Because of the mostly stochastic nature of the dining philosopher problem, given by the unordered sequence of chopstick acquisitions by the philosophers, it is possible that each participant in the simulation run is waiting indefinitely for his turn to eat. A fitting example would be the case where both neighbors of the same philosopher always take the chopsticks first. Long enough waiting times can lead to starvation, which for the dining philosophers problem is considered to have failed if even one philosopher starves. In our model, starvation does not directly imply death or termination; it refers to unfair scheduling where a philosopher may wait indefinitely while others continue to eat. In the current baseline implementation biased anyone can attempt to grab free chopsticks and fast or lucky philosophers may finish thinking earlier, retry more aggressively or reaquire chopsticks repeatedly.

This possibility of indefinite waiting times requires the addition of a mechanism that ensures fairness between the members for acquiring the chopsticks.

\subsubsection{Fairness Mechanism}

In order to combat starvation, the model is using an aging-based scheduling strategy in which hungry philosophers are dynamically prioritized by the time elapsed since their last eating event. 

\subsection{Scenario Variations}

4. explain complex scenarios we are further inspecting

\subsection{Evaluation Methodology}

In order to draw reliable conclusions, we do not rely on a single simulation run. For this reason, all simulations are done using a Monte Carlo Simulation. Depending on the scenario and parameter setting, each configuration is repeated between 50 and 100 times.

The evaluation is primarily based on summary statistics across runs, including the mean and standard deviation of measured performance and fairness metrics. For instance, the total number of eating events, the average number of eats per philosopher, as well as dispersion measures such as the minimum and maximum number of eats across philosophers are reported. Listing~\ref{lst:reportoutput} shows an example report generated from a Monte Carlo simulation. The focus of this Listing is on the structure of the reported metrics rather than on the specific numerical values.

\begin{lstlisting}[language={},keywordstyle=\color{black},commentstyle=\color{black},stringstyle=\color{black},caption={Example Report Output},
 label=lst:reportoutput]  
Runs: 100
Avg total eats: 47.96 +/- 6.62
Avg eats/philosopher: 9.59 +/- 1.32
Avg min/max eats: 7.51 / 11.79
Avg Gini (fairness): 0.088
Deadlock runs: 17/100
Avg first deadlock time: 76.99
Per-philosopher avg eats: [9.34, 9.54, 9.61, 9.75, 9.72]
Per-philosopher std eats: [2.11, 2.35, 2.04, 2.14, 1.91]
Per-philosopher avg thinking time: [49.74, 48.37, 48.97, 50.78, 49.28]
Per-philosopher avg hungry time:   [43.5, 43.47, 41.51, 41.06, 43.68]
Per-philosopher avg eating time:   [26.76, 28.16, 29.53, 28.16, 27.04]
Per-philosopher std thinking time: [12.37, 11.51, 12.69, 11.14, 12.09]
Per-philosopher std hungry time:   [11.61, 12.65, 12.81, 12.0, 11.4]
Per-philosopher std eating time:   [7.54, 8.03, 8.26, 8.34, 7.52]
Zero-eat rate by philosopher: [0.0, 0.0, 0.0, 0.0, 0.0]
\end{lstlisting}    

To monitor simulation progress and to enable debugging before running large Monte Carlo batches, we use logging during individual runs. Listing~\ref{lst:logging} illustrates a short example from the event log of an individual simulation run. 

\begin{lstlisting}[language={},keywordstyle=\color{black},commentstyle=\color{black},stringstyle=\color{black},caption={Snippet of an Example Simulation Logging},
 label=lst:logging]  
[t=0.27] Philosopher 0 is now HUNGRY
[t=0.38] Philosopher 0 is now EATING (count: 1)
[t=0.46] Philosopher 1 is now HUNGRY
[t=0.57] Philosopher 3 is now HUNGRY
[t=1.01] Philosopher 3 is now EATING (count: 1)
\end{lstlisting}  

\subsection{Parametrisation (from template)}
It is important to distinguish between parameters and parameter values. The latter should be specified e.g. using a parameter table such as 
Table \ref{tab:ParA}. Do not forget to explain where you got your parameter values from.
\begin{table}[!h]
{\small%
\newcommand{\mc}[3]{\multicolumn{#1}{#2}{#3}}
\begin{center}
\begin{tabular}{|c|c|c|c|c|c|c|c|c|c|c|}
 \hline
 $t_{end}$ [s] & $X_0$ [m]    & $V_0$ [m/s]  & $d$ & relTol & absTol & Refine & maxdist & Tol\\
 \hline              
      100      & $\binom{0}{2}$ & $\binom{2}{0}$ &  1  & 1e-3   &  1e-6  &    8   &     2   &  $\binom{1e-3}{1e-6}$ \\
 \hline
\end{tabular}
\end{center}
}%
\caption{Parameter set A.}
\label{tab:ParA}
\end{table}

\subsection{Implementation}

maybe talk about how we implemented it in general + some details maybe


Implementation should be distinguished from the conceptual modelling (see Section \ref{sec:modelling}) and described in a separate section. Keep it brief \textbf{do not} put the whole source-code into the documentation. If there are important snippets, you may use e.g. \verb|\lstinputlisting|.

 \lstinputlisting[
 language=Matlab,
 caption={Function polar2cartesian.},
 label={SRC:polar2cartesian}]{./code/BeispielCode.m}

Listing \ref{SRC:polar2cartesian} shows the source-code of \verb|polar2cartesian| in MATLAB. 
 
Source code without file can be included via:
\begin{lstlisting}[
 basicstyle=\footnotesize, % the size of the fonts that are used for the code
 frame=single,          
 language=c++,
 numbers=left,             % where to put the line-numbers
 escapeinside={//*}{*//},  % for line labeling
 caption={Definition der Klasse \texttt{InOutputVector}.},
 label=src:InOutputVector_defs]  
 class InOutputVector:public std::vector<InOutput> {
   public:
     int untreated_entry_changes;
     
     InOutputVector() {            //* \label{lnbr:InOutput_constructor} *//
       untreated_entry_changes = 0;
     }    
     void setAt(int c,double val,double t) {
       if(true==(*this)[c].already_treated) {
         untreated_entry_changes++;
       } 
       (*this)[c].set(val,t);
     }
     double* treatAt(int c,double val) {
       if(false==(*this)[c].already_treated) {
         untreated_entry_changes--;
       }
       return((*this)[c].treat());
     }
     void treatAll() {
       for(int i=0; i<this->size(); i++){
         (*this)[i].already_treated = true;
       }
       untreated_entry_changes=0;
     }
 };
\end{lstlisting}    

In row \ref{lnbr:InOutput_constructor} in Listing \ref{src:InOutputVector_defs} we find the constructor of class
\verb|InOutputVector|.

\section{Simulation Results}

Talking about results (in a storytelling way, so beginning with basic model, then extending with deadlock detection and so on, as in the notebook) + of course showing some plots


We document the results of the simulation as far as possible without potentially subjective interpretation. This is usually done with the help of plots.

\section{Discussion}
\paragraph{Summary.} We tell what we told, and summarise that we showed some important \LaTeX commands for scientific writing and displayed some dummy simulation results.
\paragraph{Result Interpretation}
We interpret the results and put them in contrast with each other and the real system. 
\paragraph{Conclusion.} We conclude that writing a proper project documentation is not so hard and hope you have fun doing it yourself. 
\paragraph{Outlook.} However, we understand that further research might be necessary.
\newpage

\bibliographystyle{plain}
\bibliography{references}

\appendix
\section{Appendix}
\subsection{Not Quite Relevant Enough}
Some stuff which is not quite important enough for the main text.
\end{document}