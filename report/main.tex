\documentclass[a4paper,11pt]{scrartcl}

\usepackage[english]{babel}
% uncomment or modify according to your operating system
% \usepackage[applemac]{inputenc} % european characters can be used (Mac OS)
%\usepackage[latin1]{inputenc} 
%\usepackage{ucs} % fuer deutsche Sonderzeichen unter Linux
%\usepackage[utf8x]{inputenc} %fuer deutsche Sonderzeichen unter Linux
\usepackage[utf8]{inputenc} %fuer deutsche Sonderzeichen unter Windows
\usepackage[T1]{fontenc}
\usepackage{graphicx} 
\usepackage{fullpage}
\usepackage{latexsym}
\usepackage{amssymb}
\usepackage{amsmath}
\usepackage{ifthen}
\usepackage{listings}
\usepackage{color}
\usepackage{hyperref}
\usepackage{cite}

\definecolor{dkgreen}{rgb}{0,0.6,0}
\definecolor{gray}{rgb}{0.5,0.5,0.5}
\definecolor{mauve}{rgb}{0.58,0,0.82}
\lstset{ %
  language=Matlab,                % the language of the code
  basicstyle=\small,              % the size of the fonts that are used for the code
  numbers=left,                   % where to put the line-numbers
  numberstyle=\tiny\color{gray},  % the style that is used for the line-numbers
  stepnumber=1,                   % the step between two line-numbers. If it's 1, each line 
                                  % will be numbered
  numbersep=5pt,                  % how far the line-numbers are from the code
  backgroundcolor=\color{white},      % choose the background color. You must add \usepackage{color}
  showspaces=false,               % show spaces adding particular underscores
  showstringspaces=false,         % underline spaces within strings
  showtabs=false,                 % show tabs within strings adding particular underscores
  frame=single,                   % adds a frame around the code
  rulecolor=\color{black},        % if not set, the frame-color may be changed on line-breaks within not-black text (e.g. commens (green here))
  tabsize=2,                      % sets default tabsize to 2 spaces
  captionpos=b,                   % sets the caption-position to bottom
  breaklines=true,                % sets automatic line breaking
  breakatwhitespace=false,        % sets if automatic breaks should only happen at whitespace
  title=\lstname,                   % show the filename of files included with \lstinputlisting;
                                   % also try caption instead of title
  keywordstyle=\color{blue},          % keyword style
  commentstyle=\color{dkgreen},       % comment style
  stringstyle=\color{mauve},         % string literal style
  escapeinside={\%*}{*)},            % if you want to add LaTeX within your code
  morekeywords={end,sortrows}               % if you want to add more keywords to the set
}


% Abkuerzungen fuer haeufige Befehle wie z.B. griechische Buchstaben
\newcommand{\N}{\mathbb{N}}
\newcommand{\Z}{\mathbb{Z}}
\newcommand{\R}{\mathbb{R}}
\newcommand{\C}{\mathbb{C}}
\newcommand{\F}{\mathcal{F}}
\newcommand{\La}{\mathcal{L}}
\newcommand{\de}{\delta}
\newcommand{\e}{\varepsilon}
\newcommand{\la}{\lambda} 
\newcommand{\p}{\varphi} 
\newcommand{\al}{\alpha} 
\newcommand{\be}{\beta} 
\newcommand{\om}{\omega}
\newcommand{\Om}{\Omega}
\newcommand{\ta}{\tau}
\newcommand{\g}{\gamma}
\newcommand{\HE}{\mathbb{H}}
\newcommand{\E}{\mathbb{E}}
\newcommand{\vG}{\varGamma}
\newcommand{\s}{\sigma}

\begin{document}

\subject{194.076 Modeling and Simulation}
\title{Discrete Event Simulation Case Study: Dining Philosophers
Problem}

%  TODO: add names
%  TODO: add description of which parts did woh in the footnote as in the template!

\publishers{Supervisor: Madlen Martinek}
\author{Marianne Musterfrau, Matricular Number - Curriculum Number: \footnote{TODO: ADD DESCRIPTION}\\
Max Mustermann ,Matricular Number - Curriculum Number: \footnote{TODO: ADD DESCRIPTION}\\
Hasan Hüseyin Günes, 12229237 - Curriculum Number: UE 066 645\footnote{TODO: ADD DESCRIPTION}\\
}

\maketitle
% TODO
\begin{abstract}
TODO: Brief Abstract
\end{abstract}

\newpage

\tableofcontents

\newpage

\section{Introduction}

When multiple processes or systems commonly use a limited amount of resource, many potential problems can occur. If the coordination is done poorly, the whole process can get stuck because of a deadlock, where the progress of the system is blocked. Besides, without a certain fairness given by the distributed system, individual components can have major waiting times compared to others, also called starvation. The Dining Philosophers problem\footnote{\url{https://en.wikipedia.org/wiki/Dining_philosophers_problem}} is a simple but realistic example addressing such situations. It helps to understand why these issues can happen and gives a good base concept for inspecting outputs of different methods in order to prevent such issues. 

In the Dining Philosophers problem, philosophers are sitting around a table with one chopstick placed between two philosophers. Over time, each philosopher repeats the cycle of thinking, being hungry and eating. To eat, a philosopher needs two chopsticks (left and right chopstick) next to him/her, whereas one chopstick can only be used by one philosophers at a time.

\paragraph{Reated Work.} This problem has also been addressed by \cite{zeigler2000theory}, for example, but it differs greatly from the current project. Note, that literature must be inserted into File \textit{references.bib} and can be cited via \verb|\cite{QUELLE}|.
\paragraph{Aim.} However, goal of this project is to reduce this problem with the help of a template.

\section{Model}
\subsection{Modelling}
\label{sec:modelling}
We first describe the conceptual modeling. Hereby we may apply various types of formulas such as numbered equations, like
\begin{equation}
\label{eq:pythogoras}
 a^2 + b^2 = c^2,
\end{equation}
or non-numbered equations like
\begin{equation*}
 \al + \be + \g = 180.
\end{equation*}
Numberes equations can be referenced via (\ref{eq:pythogoras}).

Use e.g. \textit{eqnarray}, \textit{align} or \textit{multline} environments for multi-row equations, such as
\begin{eqnarray}
  a &=& b\cdot \frac{sin(\al)}{sin(\be)} \\
    &=& c\cdot \frac{sin(\al)}{sin(\g)}.
\end{eqnarray}

We write a matrix using the \textit{matrix} command.
\[ r\cdot
\begin{pmatrix}
 cos(\p) & sin(\p) & 0 \\
 -sin(\p) & cos(\p) & 0 \\
 0 & 0 & 1
\end{pmatrix}.
\]

It is always good to use pictures and charts for visualising the model structure, such as Figure \ref{fig:figure}. Do not forget to refer to a figure in the main text.
\begin{figure}[!ht]
 \centering
 \includegraphics[width=\textwidth]{./images/ClassEAmplifier.pdf}
 % bBheightode23last5Sec.png: 1130x487 pixel, 90dpi, 31.89x13.75 cm, bb=0 0 904 390
 \caption{This is a sample image if a Class-E Amplifier.}
 \label{fig:figure}
\end{figure}
\subsection{Parametrisation}
It is important to distinguish between parameters and parameter values. The latter should be specified e.g. using a parameter table such as 
Table \ref{tab:ParA}. Do not forget to explain where you got your parameter values from.
\begin{table}[!h]
{\small%
\newcommand{\mc}[3]{\multicolumn{#1}{#2}{#3}}
\begin{center}
\begin{tabular}{|c|c|c|c|c|c|c|c|c|c|c|}
 \hline
 $t_{end}$ [s] & $X_0$ [m]    & $V_0$ [m/s]  & $d$ & relTol & absTol & Refine & maxdist & Tol\\
 \hline              
      100      & $\binom{0}{2}$ & $\binom{2}{0}$ &  1  & 1e-3   &  1e-6  &    8   &     2   &  $\binom{1e-3}{1e-6}$ \\
 \hline
\end{tabular}
\end{center}
}%
\caption{Parameter set A.}
\label{tab:ParA}
\end{table}

\subsection{Implementation}

Implementation should be distinguished from the conceptual modelling (see Section \ref{sec:modelling}) and described in a separate section. Keep it brief \textbf{do not} put the whole source-code into the documentation. If there are important snippets, you may use e.g. \verb|\lstinputlisting|.

 \lstinputlisting[
 language=Matlab,
 caption={Function polar2cartesian.},
 label={SRC:polar2cartesian}]{./code/BeispielCode.m}

Listing \ref{SRC:polar2cartesian} shows the source-code of \verb|polar2cartesian| in MATLAB. 
 
Source code without file can be included via:
\begin{lstlisting}[
 basicstyle=\footnotesize, % the size of the fonts that are used for the code
 frame=single,          
 language=c++,
 numbers=left,             % where to put the line-numbers
 escapeinside={//*}{*//},  % for line labeling
 caption={Definition der Klasse \texttt{InOutputVector}.},
 label=src:InOutputVector_defs]  
 class InOutputVector:public std::vector<InOutput> {
   public:
     int untreated_entry_changes;
     
     InOutputVector() {            //* \label{lnbr:InOutput_constructor} *//
       untreated_entry_changes = 0;
     }    
     void setAt(int c,double val,double t) {
       if(true==(*this)[c].already_treated) {
         untreated_entry_changes++;
       } 
       (*this)[c].set(val,t);
     }
     double* treatAt(int c,double val) {
       if(false==(*this)[c].already_treated) {
         untreated_entry_changes--;
       }
       return((*this)[c].treat());
     }
     void treatAll() {
       for(int i=0; i<this->size(); i++){
         (*this)[i].already_treated = true;
       }
       untreated_entry_changes=0;
     }
 };
\end{lstlisting}    

In row \ref{lnbr:InOutput_constructor} in Listing \ref{src:InOutputVector_defs} we find the constructor of class
\verb|InOutputVector|.

\section{Simulation Results}
We document the results of the simulation as far as possible without potentially subjective interpretation. This is usually done with the help of plots, as in Figure \ref{fig:figure2}.
\begin{figure}[!ht]
 \centering
 \includegraphics[width=\textwidth]{./images/bBheightode23last5Sec.png}
 % bBheightode23last5Sec.png: 1130x487 pixel, 90dpi, 31.89x13.75 cm, bb=0 0 904 390
 \caption{Simulation results when using parameter set A (see Table \ref{tab:ParA})}
 \label{fig:figure2}
\end{figure}

\section{Discussion}
\paragraph{Summary.} We tell what we told, and summarise that we showed some important \LaTeX commands for scientific writing and displayed some dummy simulation results.
\paragraph{Result Interpretation}
We interpret the results and put them in contrast with each other and the real system. 
\paragraph{Conclusion.} We conclude that writing a proper project documentation is not so hard and hope you have fun doing it yourself. 
\paragraph{Outlook.} However, we understand that further research might be necessary.
\newpage

\bibliographystyle{plain}
\bibliography{references}

\appendix
\section{Appendix}
\subsection{Not Quite Relevant Enough}
Some stuff which is not quite important enough for the main text.
\end{document}